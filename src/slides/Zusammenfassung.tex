\section*{Zusammenfassung}

\begin{frame}{Zusammenfassung}
    \begin{itemize}
        \item Gefangenen-Dilemma (PD) kann iteriert werden (IPD)
        \item Memory-Länge bei Strategien im IPD ist egal (für Memory $> 0$)
        \item IPD modellierbar als diskrete, homogene Markov-Kette mit Markov-Matrix
        \item Existenz von zero-determinant (ZD) Strategien im IPD: Lineare Abhängigkeit der Ergebnisse zweier Strategien
        \item Ausprägungen von ZD-Strategien: Forcierung des Gegner-Scores $P \leq \text{score}(p_2) \leq R$, Erpressung mit Faktor $\chi$
        \item Theory of Mind beider Spieler hat Einfluss auf Erfolg von ZD-Strategien
    \end{itemize}
\end{frame}