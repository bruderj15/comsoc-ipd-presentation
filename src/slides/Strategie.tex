\section{Strategien}

\begin{frame}{Modellierung}
    \begin{itemize}
        \item Abwandlung Wahlfunktion $w: P \rightarrow W$ zu Strategie $s^{p_i}: G^1 \rightarrow W$ \\
        $\Rightarrow$ $s^{p_i}_k(G^1_{k-1}) = \ldots$ \\
        $\Rightarrow$ Zugriff auf Ergebnis letzter Runde: Auslegung des Spiels ist \textbf{Memory-One}
        $\Rightarrow$ \textit{Spielen} einer Runde $G_k$ ist Zustandsübergang in Markov-Kette
        \item Spiel und Auszahlungsmatrix ändern sich nicht: Strategien $s^{p_1}, s^{p_2}$ implizieren Markov-Matrix
        \item Daher: Strategie $s^{p_i}$ ausdrückbar als Vektor $v_{s^{p_i}} = (P_{(C, C)}, P_{(C, D)}, P_{(D, C)}, P_{(D, D)})$,
        wobei $P_{W \times W}$ ist Wahrscheinlichkeit, dass Spieler $p_i$ in Runde $k$ wegen Ergebnis($G^1_{k-1}$) $ = (W, W)$ \textbf{kooperiert}
    \end{itemize}
\end{frame}

\begin{frame}{Beispiele}
    \begin{itemize}
        \item All-Defect: $(0, 0, 0, 0)$
        \item All-Cooperate: $(1, 1, 1, 1)$
        \item Tit-For-Tat: $(1, 0, 1, 0)$
        \item Generous Tit-For-Tat: $(1, q, 1, q)$ mit $0 < q \leq \frac{3}{10}$
        \item Win-Stay Lose-Shift: $(1, 0, 0, 1)$
        \item Random: $(\frac{1}{2}, \frac{1}{2}, \frac{1}{2}, \frac{1}{2})$
    \end{itemize}
\end{frame}