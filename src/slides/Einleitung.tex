\section*{Einleitung}

\begin{frame}{Beispiel}
    \begin{itemize}
        \item Getrenntes, zeitgleiches Verhör zweier Verdächtiger, die gegeneinander aussagen \textit{können}
        \item Je nach Aussagen ergeben sich verschiedene Haftstrafen
        \item Jeder Verdächtige sagt gegen den anderen Verdächtigen aus - oder schweigt
        \item Mögliche Haftstrafen:
        \begin{itemize}
            \item Beide schweigen: jeweils 5 Jahre
            \item Einer schweigt, anderer sagt aus: Schweigender 10 Jahre, Aussagender 2 Jahre
            \item Beide sagen aus: jeweils 10 Jahre
        \end{itemize}
        \item Visualisierung durch Haftstrafenmatrix (Tafel)
    \end{itemize}
\end{frame}

\begin{frame}{Motivation}
    \begin{itemize}
        \item Die Verhörer behandeln beide Verdächtige gleich
        \item Beide Verdächtige wissen über das zeitgleiche Verhör
        \item Beide Verdächtige kennen alle möglichen Ausgänge der Verhöre
        \item Beide Verdächtige haben die gleiche Wahl: Aussagen oder Schweigen
        \item Beide Verdächtige werden nach gleichem Maß bestraft - je nach Aussage
        \item \textbf{Das Verhör ist also fair?}
        \item \textbf{Welche Wahl ist die Beste aus Sicht eines Verdächtigen?}
        \item \textbf{Was passiert, wenn wiederholt gegeneinander ausgesagt wird?}
    \end{itemize}
\end{frame}
