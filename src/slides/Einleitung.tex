\section*{Einleitung}

\begin{frame}{Beispiel}
    \begin{itemize}
        \item Getrenntes, zeitgleiches Verhör zweier Verdächtiger, die gegeneinander aussagen \textit{können}
        \item Je nach Aussagen ergeben sich verschiedene Haftstrafen
        \item Jeder Verdächtige sagt gegen den anderen Verdächtigen aus - oder schweigt
        \item Mögliche Haftstrafen:
        \begin{itemize}
            \item Beide schweigen: jeweils 5 Jahre
            \item Einer schweigt, anderer sagt aus: Schweigender 20 Jahre, Aussagender 2 Jahre
            \item Beide sagen aus: jeweils 15 Jahre
        \end{itemize}
        \item Visualisierung durch Haftstrafenmatrix (Tafel)
    \end{itemize}
\end{frame}

\begin{frame}{Motivation: Fragestellungen}
    \begin{itemize}
        \item Nash-Equilibrium bei einmaligem Verhör: Aussagen beider
        \item Was passiert, wenn das Verhör mehrmals (unendlich oft?) wiederholt wird?
        \item Welche Strategie (Aussage oder nicht) wählt ein Verdächtiger?
        \item Hat ein Verdächtiger einen strategischen Vorteil, wenn er sich die Ergebnisse aller vergangenen Verhöre beider Verdächtiger merkt?
        \item Kann ein Verdächtiger den anderen manipulieren?
        \item Kann ein Verdächtiger den anderen erpressen?
    \end{itemize}
\end{frame}

\begin{frame}{Motivation: Anwendung}
    \begin{itemize}
        \item Anwendung in evolutionärer Biologie \parencite{axelrod-hamilton-1981}
        \begin{itemize}
            \item Symbionten und Parasiten
            \item Altruismus und Nepotismus
            \item Aber: Meist in oder zwischen Populationen, IPD (hier) nur zwischen zwei Individuen
        \end{itemize}
        \item In der Ökonomie \parencite{honhon-hyndman-2020}
        \begin{itemize}
            \item Preiskämpfe
            \item (Tarif-)Verhandlungen
            \item Partnerschaften
        \end{itemize}
        \item In der HTWK-Mensa: Verhandlung der Portionsgröße bei der Essensausgabe
    \end{itemize}
\end{frame}

\begin{frame}{Wiederholung: Nash-Equilibrium}
    \begin{itemize}
        \item Jeder Spieler $p_i \in P, i \in \mathbb{N}$ spielt eine Strategie $s_{p_i}$
        \item Strategieprofil $\left( s_{p_i}, \ldots, s_{p_n} \right)$ ist n-Tupel der Strategien aller $n \in \mathbb{N}_{\geq 2}$ Spieler
        \item Strategieprofil bildet Nash-Equilibrium genau dann, wenn kein Spieler durch alleiniges Ändern seiner Strategie einen Vorteil erhalten würde
        \item Hier: Individuelle Strategien sind beliebig (dominant oder pur oder gemischt)
    \end{itemize}

\end{frame}
