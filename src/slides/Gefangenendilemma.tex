\section{Gefangenendilemma}

\begin{frame}{Modellierung (PD)}
    \begin{itemize}
        \item Spiel $g \in G$ mit Spielern $p_1, p_2 \in P$
        \item Wahlmöglichkeit $W = \left\{ C \text{ (Cooperate)}, D \text{ (Defect)} \right\}$
        \item Wahlfunktion $w: P \rightarrow W$
        \item Ergebnis($g$) $ \in W \times W = \left\{ (C, C), (C, D), (D, C), (D, D) \right\}$ mit \\
            Ergebnis($g$) $ = (w(p_1), w(p_2))$
        \item Vier abstrakte Kenngrößen:
        \begin{itemize}
            \item Reward $R \in \mathbb{R}$ ((C, C))
            \item Temptation $T \in \mathbb{R}$ ((C, \textbf{D})/(\textbf{D}, C))
            \item Punishment $P \in \mathbb{R}$ ((D, D))
            \item Sucker's Payoff $S \in \mathbb{R}$ ((\textbf{C}, D)/(D, \textbf{C}))
        \end{itemize}
        \item Es gilt: $T > R > P > S \land 2R > T + S$, erzeugt Auszahlungsmatrix (Tafel)
    \end{itemize}
\end{frame}

\begin{frame}{Modellierung: Erweiterung um Iterationen (IPD)}
    \begin{itemize}
        \item Einfaches Spiel $g \in G$ wird $n \in \mathbb{N}^+$ mal iteriert: $g_n \in G^n$
        \item Jede Iteration (Runde) $r \in \mathbb{N}^+_{\leq n}$ ist selbst (einfaches) Spiel $g_1 \in G^1$
        \item Ergebnis($g_n$) $ = \left[ \text{Ergebnis}(g_{1_1}), \ldots , \text{Ergebnis}(g_{1_n}) \right]$
        \item $n \in \mathbb{N}^+$ ist beliebig, aber fest und unbekannt für $p_1, p_2$
        \item Warum? Wenn $n$ bekannt: Nash-Equilibrium bei $w(\_) = C$ (Induktion, Tafel)
    \end{itemize}
\end{frame}